\documentclass[a4paper,12pt]{report}
\errorcontextlines=999

\usepackage[utf8]{inputenc}
\usepackage[T1]{fontenc}

\usepackage{amsmath}
\usepackage{amssymb}
\usepackage{amsthm}

\newtheorem{definition}{Definition}
\newtheorem{fact}{Fact}
\newtheorem{theorem}{Theorem}
\newtheorem{lemma}{Lemma}

\usepackage{graphicx}
\usepackage{caption}
\usepackage{subcaption}
\usepackage{csquotes}
\usepackage{hyperref}
\usepackage[nameinlink,capitalise]{cleveref}
\newcommand{\argmin}{\operatorname*{arg\,min}}

\crefname{definition}{definition}{definitions}
\Crefname{definition}{Definition}{Definitions}

\crefname{fact}{fact}{facts}
\Crefname{fact}{Fact}{Facts}

\crefname{theorem}{theorem}{theorems}
\Crefname{theorem}{Theorem}{Theorems}

\crefname{lemma}{lemma}{lemmas}
\Crefname{lemma}{Lemma}{Lemmas}

\title{A Ring that is a PID but not an ED}
\author{Bajtek Rogue}
\date{\today}

\begin{document}
\maketitle

\section*{Motivation}

When studying rings, we encounter the fact that all Euclidean domains are principal ideal domains. A natural question arises: is the converse true? In this paper, we prove that it is not. Specifically, the converse fails for the ring $\mathbb{Z}\left[\frac{1+\sqrt{-19}}{2}\right]$. To establish this result, we require two useful lemmas.

\section*{Lemmas}

\begin{lemma}\label{lem:dh_norm}
    Let $R$ be a ring. A function $N:R\to\mathbb{N}$ is called a \textbf{Dedekind--Hasse norm} if 
    \[
        \forall a,b\in R\setminus\{0\} \quad b\nmid a\implies \exists s,t\in R\quad 0<N(sa-tb)<N(b).
    \]
    If $R$ has a Dedekind--Hasse norm, then $R$ is a principal ideal domain.
\end{lemma}
\begin{proof}
    Suppose $R$ is a ring with Dedekind--Hasse norm $N:R\to\mathbb{N}$. Let $I\triangleleft R$ be any ideal of $R$. Take $g=\argmin_{i\in I\setminus\{0\}} N(i)$. Since $\varnothing\neq \{N(i):i\in I\setminus\{0\}\}\subseteq\mathbb{N}$, such a $g$ exists by the well-ordering principle on $\mathbb{N}$. Now let $i\in I$.
    
    \medskip
    \noindent\textbf{Case 1:} $g\mid i$.
    
    Then $i=gx$ for some $x\in R$, and thus $i\in \langle g \rangle$.
    
    \medskip
    \noindent\textbf{Case 2:} $g\nmid i$.
    
    Since $N$ is a Dedekind--Hasse norm, there exist $s,t\in R$ such that $0<N(si-tg)<N(g)$.
    We have $si-tg\in \langle i,g \rangle \subseteq I$ since $i\in I$, so $si-tg\in I$. But this contradicts the minimality of the norm of $g$, so this case is impossible.
    
    \medskip
    Thus $i\in \langle g \rangle$, but $i$ was arbitrary, so $I\subseteq\langle g \rangle$. Since $g\in I$, we have $\langle g\rangle\subseteq I$, and therefore $I=\langle g\rangle$. Hence $R$ is a principal ideal domain.
\end{proof}

\begin{lemma}\label{lem:usd}
    Let $R$ be a ring. An element $u\in R\setminus (R^*\cup\{0\})$ is called a \textbf{universal side divisor} if
    \[
        \forall x\in R\;\exists y\in R^*\cup\{0\}\quad u\mid x-y.
    \]
    If $R$ is a Euclidean domain, then $R$ has a universal side divisor. 
\end{lemma}
\begin{proof}
    Suppose $R$ is a Euclidean domain with norm $N:R\to\mathbb{N}$. Take $u = \displaystyle\argmin_{a \in R \setminus (R^* \cup \{0\})} N(a)$. Since $\varnothing\neq\{N(a):a\in R\setminus (R^*\cup\{0\})\}\subseteq \mathbb{N}$, such a $u$ exists by the well-ordering principle on $\mathbb{N}$. Let $x\in R$. By the division algorithm, we can write $x=uq+r$ where $q,r\in R$.
    
    \medskip
    \noindent\textbf{Case 1:} $r=0$.
    
    Then $u\mid x$, so $u\mid x-0$. Since $0\in R^*\cup\{0\}$, we have that $u$ is a universal side divisor.
    
    \medskip
    \noindent\textbf{Case 2:} $r\neq 0$.
    
    Then $N(r)<N(u)$, and by the minimality of $N(u)$, we know that $r\in R^*\cup\{0\}$. Moreover, $u\mid x-r$, so $u$ is a universal side divisor.
\end{proof}

\section*{Constructing a Dedekind--Hasse Norm}

Consider the ring $\mathbb{Z}\left[\frac{1+\sqrt{-19}}{2}\right]$ and the function $N:\mathbb{Z}\left[\frac{1+\sqrt{-19}}{2}\right]\to\mathbb{N}$ given by
\[
    N\left(a+b\frac{1+\sqrt{-19}}{2}\right)=\left(a+b\frac{1+\sqrt{-19}}{2}\right)\cdot\left(a+b\frac{1-\sqrt{-19}}{2}\right).
\]
We can easily verify that $N\left(a+b\frac{1+\sqrt{-19}}{2}\right)=a^2+ab+5b^2$. Since $N$ is the norm map from complex numbers, $N$ is multiplicative. Moreover, $N(a+b\sqrt{-19})=a^2+19b^2$. 

If $n,m\in\mathbb{Z}$, then $n+m\sqrt{-19}\in\mathbb{Z}\left[\frac{1+\sqrt{-19}}{2}\right]$ since 
\[
n+m\sqrt{-19}=(n-m)+2m\cdot \frac{1+\sqrt{-19}}{2}.
\]

Take $\alpha,\beta\in \mathbb{Z}\left[\frac{1+\sqrt{-19}}{2}\right]$ with $\alpha,\beta\neq 0$ and $\beta\nmid\alpha$. Thus $\frac{\alpha}{\beta}\notin\mathbb{Z}\left[\frac{1+\sqrt{-19}}{2}\right]$. However, $\frac{\alpha}{\beta}\in\mathbb{Q}(\sqrt{-19})$ since $\mathbb{Q}(\sqrt{-19})$ is the smallest field containing $\mathbb{Z}\left[\frac{1+\sqrt{-19}}{2}\right]$. Hence we can write $\frac{\alpha}{\beta}=\frac{a+b\sqrt{-19}}{c}$ where $a,b,c\in\mathbb{Z}$. Without loss of generality, we may assume that $\gcd(a,b,c)=1$ and $c>0$. Also, $c\ge 2$ since if $c=1$, then $\frac{\alpha}{\beta}\in\mathbb{Z}\left[\frac{1+\sqrt{-19}}{2}\right]$, which contradicts our assumption. 

Note that the condition $0<N(s\alpha-t\beta)<N(\beta)$ is equivalent to $0<N\left(s\cdot\frac{\alpha}{\beta}-t\right)<N(1)=1$ by the multiplicativity of $N$. 

\subsection*{Case 1: $c\geq 6$}

By Bézout's identity, we can write $ax+by+cz=1$ for some $x,y,z\in\mathbb{Z}$. Now perform division with remainder in integers of $ay-19bx$ by $c$ to get $ay-19bx=qc+r$, where $r$ is chosen so that $|r|\le\frac{c}{2}$. Put $s=y+x\sqrt{-19}$ and $t=q-z\sqrt{-19}$. We obtain
\begin{align*}
    s\cdot\frac{\alpha}{\beta}-t
    &=(y+x\sqrt{-19})\cdot \frac{a+b\sqrt{-19}}{c}-(q-z\sqrt{-19}) \\
    &=\frac{(ay-19bx)+(ax+by)\sqrt{-19}}{c}+\frac{-qc+cz\sqrt{-19}}{c}\\ 
    &=\frac{(ay-19bx-qc)+(ax+by+cz)\sqrt{-19}}{c}.
\end{align*}
Recall that $ax+by+cz=1$ and $ay-19bx-qc=r$. We get $s\cdot\frac{\alpha}{\beta}-t=\frac{r+\sqrt{-19}}{c}$, and
\[
    N\left(\frac{r+\sqrt{-19}}{c}\right)=\frac{r^2+19}{c^2}\le\frac{\frac{c^2}{4}+19}{c^2}=\frac{1}{4}+\frac{19}{c^2}.
\]
If $c\ge 6$, then $\frac{1}{4}+\frac{19}{c^2}<1$, so $N$ satisfies the Dedekind--Hasse norm property. It remains to check the values $c\in\{2,3,4,5\}$.

\subsection*{Case 2: $c=2$}

Let $c=2$. Then $a\not \equiv b\pmod{2}$, because if $a\equiv b\equiv 0\pmod{2}$, then $\frac{\alpha}{\beta}=\frac{a+b\sqrt{-19}}{2} \in\mathbb{Z}\left[\frac{1+\sqrt{-19}}{2}\right]$. Put $s=1$ and $t=\frac{a-b-1}{2}+b\cdot \frac{1+\sqrt{-19}}{2}$. Note that $\frac{a-b-1}{2}\in\mathbb{Z}$ because $a\not \equiv b\pmod{2}$. We get
\[
    s\cdot\frac{\alpha}{\beta}-t=1\cdot\frac{a+b\sqrt{-19}}{2}-\left(\frac{a-b-1}{2}+b\cdot \frac{1+\sqrt{-19}}{2}\right)=\frac{1}{2},
\]
and $0<N\left(\frac{1}{2}\right)=\frac{1}{4}<1$, as needed.

\subsection*{Case 3: $c=3$}

Let $c=3$. If $a\equiv b\equiv 0\pmod{3}$, then $\gcd(a,b,c)=3>1$, which is impossible. Recall that $\forall n\in\mathbb{Z}$, we have $n^2\equiv 0$ or $1\pmod{3}$. Thus $a^2+b^2\equiv 0\pmod{3}$ if and only if $a\equiv b\equiv 0\pmod{3}$. But we know this is not true, so $a^2+b^2\not\equiv 0\pmod{3}$. Also, $a^2+19b^2\equiv a^2+b^2\pmod{3}$, so $a^2+19b^2\not\equiv 0\pmod{3}$. 

Now perform division with remainder in integers of $a^2+19b^2$ by $3$ to get $a^2+19b^2=3q+r$, where $r\in\{1,2\}$ by the previous argument. Put $s=a-b\sqrt{-19}$ and $t=q$. We get
\[
    s\cdot\frac{\alpha}{\beta}-t=(a-b\sqrt{-19})\cdot\frac{a+b\sqrt{-19}}{3}-q=\frac{a^2+19b^2-3q}{3}=\frac{r}{3},
\]
and $0<N\left(\frac{r}{3}\right)=\frac{r^2}{9}<1$ because $r\neq 0$. Hence $N$ satisfies the Dedekind--Hasse norm property.

\subsection*{Case 4: $c=4$}

Let $c=4$. If $a\equiv b\equiv 0\pmod{2}$, then $\gcd(a,b,c)\ge2>1$, which is impossible.

\medskip
\noindent\textbf{Subcase 4.1:} $a\equiv 0\pmod{2}$ or $b\equiv 0\pmod{2}$ (but not both).

Either way, we have $a^2+19b^2\equiv 1\pmod{2}$. Now perform division with remainder in integers of $a^2+19b^2$ by $4$ to get $a^2+19b^2=4q+r$, where $r\in\{1,3\}$. Note that if $r\in\{0,2\}$, then we would have $a^2+19b^2\equiv 0\pmod{2}$, a contradiction. Put $s=a-b\sqrt{-19}$ and $t=q$. We get
\[
    s\cdot\frac{\alpha}{\beta}-t=(a-b\sqrt{-19})\cdot\frac{a+b\sqrt{-19}}{4}-q=\frac{a^2+19b^2-4q}{4}=\frac{r}{4},
\]
and $0<N\left(\frac{r}{4}\right)=\frac{r^2}{16}<1$ because $r\neq 0$.

\medskip
\noindent\textbf{Subcase 4.2:} $a\equiv b\equiv 1\pmod{2}$.

Recall that for $n\in\{1,3,5,7\}$, we have $n^2\equiv 1\pmod{8}$. Hence $a^2+19b^2\equiv 1+3\cdot1\equiv 4\pmod{8}$. Thus we have $a^2+19b^2=8q+4$. Put $s=\frac{a-b}{2}+b\cdot\frac{1+\sqrt{-19}}{2}$ and $t=q$. Note that $\frac{a-b}{2}\in\mathbb{Z}$ because $a\equiv b\equiv 1\pmod{2}$. We get
\[
    s\cdot\frac{\alpha}{\beta}-t=\left(\frac{a-b}{2}+b\cdot\frac{1+\sqrt{-19}}{2}\right)\cdot\frac{a+b\sqrt{-19}}{4}-q=\frac{a^2+19b^2-8q}{8}=\frac{4}{8}.
\]
Obviously $\frac{4}{8}=\frac{1}{2}$, and $0<N\left(\frac{1}{2}\right)=\frac{1}{4}<1$. Hence for $c=4$, the Dedekind--Hasse norm property holds.

\subsection*{Case 5: $c=5$}

Let $c=5$. Then $a\not\equiv 0\pmod{5}$ or $b\not\equiv 0\pmod{5}$, because otherwise $\gcd(a,b,c)=5>1$. Perform division with remainder in integers of $a^2+19b^2$ by $15$ to get $a^2+19b^2=15q+r$. Note that 
\[
a^2+19b^2\equiv a^2+4b^2\pmod{15}.
\]
We have $a^2+4b^2\equiv 0\pmod{15}$ if and only if $a\equiv b\equiv 0\pmod{15}$ or $(ab^{-1})^2\equiv -4\equiv 11\pmod{15}$. We know that the first case is impossible. However, for all $n\in\mathbb{Z}$, we have $n^2\equiv 0,1,4,6,9,$ or $10\pmod{15}$. Thus $r\neq 0$. 

Put $s=a-b\sqrt{-19}$ and $t=3q$. We get
\[
    s\cdot\frac{\alpha}{\beta}-t=(a-b\sqrt{-19})\cdot\frac{a+b\sqrt{-19}}{5}-3q=\frac{a^2+19b^2-15q}{5}=\frac{r}{5},
\]
and so $0<N\left(\frac{r}{5}\right)=\frac{r^2}{25}<1$ because $r\neq 0$. 

This completes all cases, so the function $N$ is a Dedekind--Hasse norm on $\mathbb{Z}\left[\frac{1+\sqrt{-19}}{2}\right]$. By \Cref{lem:dh_norm}, we conclude that $\mathbb{Z}\left[\frac{1+\sqrt{-19}}{2}\right]$ is a principal ideal domain.

\section*{No Universal Side Divisor}

Recall that since $N$ is multiplicative, if $x\mid y$, then $N(x)\mid N(y)$ for any $x,y\in \mathbb{Z}\left[\frac{1+\sqrt{-19}}{2}\right]$. Let us find the units in $\mathbb{Z}\left[\frac{1+\sqrt{-19}}{2}\right]$. Take $v\in \mathbb{Z}\left[\frac{1+\sqrt{-19}}{2}\right]^*$ with $v=a+b\cdot\frac{1+\sqrt{-19}}{2}$. We have 
\[
N(v)=a^2+ab+5b^2=\left(a+\frac{b}{2}\right)^2+\frac{19}{4}b^2\ge5
\]
if $b\neq 0$. Since $v\mid1$ and $N(1)=1$, we have $N(v)=1$. Hence $b=0$, and then $a^2=1$, which implies $a=\pm1$. Thus $\mathbb{Z}\left[\frac{1+\sqrt{-19}}{2}\right]^*=\{-1,1\}$. 

Also, we have $N(z)<5 \implies z\in\{\pm1,\pm2\}$ for $z\in \mathbb{Z}\left[\frac{1+\sqrt{-19}}{2}\right]$. Suppose $u\in \mathbb{Z}\left[\frac{1+\sqrt{-19}}{2}\right]\setminus\{-1,0,1\}$ is a universal side divisor. We have $N(u)\neq 1$. Also, there exists $y\in\{-1,0,1\}$ such that $u\mid 2-y$ since $2\in \mathbb{Z}\left[\frac{1+\sqrt{-19}}{2}\right]$.

\medskip
\noindent\textbf{Case 1:} $y=1$.

Then $u\mid 1$, so $u\in \mathbb{Z}\left[\frac{1+\sqrt{-19}}{2}\right]^*$, which is impossible.

\medskip
\noindent\textbf{Case 2:} $y=0$.

Then $u\mid 2$, so $N(u)\mid N(2)=4$. That means $N(u)=4$, which yields $u=\pm2$.

\medskip
\noindent\textbf{Case 3:} $y=-1$.

Then $u\mid 3$, so $N(u)\mid N(3)=9$. Since $N(u)\neq 3$, we have $N(u)=9$. Let $u=n+m\cdot \frac{1+\sqrt{-19}}{2}$. Then $N(u)=n^2+nm+5m^2=9$, so $3\mid n^2+nm+5m^2$. Note that $n^2+nm+5m^2\equiv 0\pmod{3}$ if and only if $n\equiv m\equiv 0\pmod{3}$. Hence we can write $n=3k$ and $m=3\ell$ for $k,\ell\in\mathbb{Z}$. The equation becomes $9k^2+9k\ell+45\ell^2=9$, and so $k^2+k\ell+5\ell^2=1$. By previous calculations, we know that $\ell=0$ and $k=\pm1$. Thus $u=\pm 3$. 

Hence the only possible universal side divisors of $\mathbb{Z}\left[\frac{1+\sqrt{-19}}{2}\right]$ are $\pm2$ and $\pm 3$. If $u$ is a universal side divisor, then so is $-u$. Thus we only need to consider $2$ and $3$. 

Take $\frac{1+\sqrt{-19}}{2}\in\mathbb{Z}\left[\frac{1+\sqrt{-19}}{2}\right]$. There must exist a $z\in\{-1,0,1\}$ such that $2\mid \frac{1+\sqrt{-19}}{2}-z$ or $3\mid \frac{1+\sqrt{-19}}{2}-z$. Thus we need $2$ or $3$ to divide one of $\frac{1+\sqrt{-19}}{2}$, $\frac{-1+\sqrt{-19}}{2}$, or $\frac{3+\sqrt{-19}}{2}$. However, $N(2)=4$ and $N(3)=9$, but
\[
    N\left(\frac{1+\sqrt{-19}}{2}\right)=5, \quad N\left(\frac{-1+\sqrt{-19}}{2}\right)=5, \quad N\left(\frac{3+\sqrt{-19}}{2}\right)=7.
\]
Since divisibility implies divisibility of norms, neither $2$ nor $3$ divides any of these elements, so they are not universal side divisors of $\mathbb{Z}\left[\frac{1+\sqrt{-19}}{2}\right]$. Since these were the only possible candidates, $\mathbb{Z}\left[\frac{1+\sqrt{-19}}{2}\right]$ has no universal side divisors. Hence it is not a Euclidean domain by \Cref{lem:usd}.

\end{document}