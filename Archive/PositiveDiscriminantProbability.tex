\documentclass[a4paper,12pt]{report}
\errorcontextlines=999

\usepackage[utf8]{inputenc}
\usepackage[T1]{fontenc}

\usepackage{amsmath}
\usepackage{amssymb}
\usepackage{amsthm}

\newtheorem{definition}{Definition}
\newtheorem{fact}{Fact}
\newtheorem{theorem}{Theorem}
\newtheorem{lemma}{Lemma}

\usepackage{graphicx}
\usepackage{caption}
\usepackage{subcaption}
\usepackage{csquotes}
\usepackage{hyperref}
\usepackage[nameinlink,capitalise]{cleveref}

\DeclareMathOperator{\Orb}{Orb}
\DeclareMathOperator{\Fix}{Fix}
\DeclareMathOperator{\Mat}{Mat}
\newcommand{\argmin}{\operatorname*{arg\,min}}

\crefname{definition}{definition}{definitions}
\Crefname{definition}{Definition}{Definitions}

\crefname{fact}{fact}{facts}
\Crefname{fact}{Fact}{Facts}

\crefname{theorem}{theorem}{theorems}
\Crefname{theorem}{Theorem}{Theorems}

\crefname{lemma}{lemma}{lemmas}
\Crefname{lemma}{Lemma}{Lemmas}

\title{Probability of Real Roots in Random Quadratic Equations}
\author{Bajtek Rogue}
\date{\today}

\begin{document}
\maketitle

\section*{Problem}

Let $ax^2+bx+c=0$ be a quadratic equation with $a,b,c\in\mathbb{R}$. 

We want to calculate the probability that it has real roots. Recall that a quadratic equation either has all real roots or all complex (non-real) roots, and the roots are real if and only if the discriminant satisfies $b^2-4ac\ge0$.

We will assume that $a,b,c$ are chosen uniformly and independently from a symmetric interval $[-M;M]$, and then take the limit as $M\to\infty$.

\section*{Solution}

Let $M>0$ and let $A,B,C$ be independent random variables with uniform distribution:
\[
A,B,C\sim\mathcal{U}([-M;M]).
\]
Put $D=B^2-4AC$. We want to calculate
\[
\lim_{M\to\infty} \mathbb{P}[D\ge0].
\]
The desired probability is
\[
\mathbb{P}[D\ge0]=\iiint_{[-M;M]^3}f_A(x)\cdot f_B(y) \cdot f_C(z)\cdot \mathbf{1}_{y^2\ge4xz}\,\mathrm{d}x\,\mathrm{d}y\,\mathrm{d}z,
\]
where $f_A$, $f_B$, and $f_C$ are the probability density functions. Recall that the probability density function for $\mathcal{U}([-M;M])$ is given by $f(t)=\frac{1}{2M}$.

\subsection*{Calculating the Integrals}

We split the integration region into two pieces based on the sign of $4xz$:
\begin{itemize}
    \item If $4xz<0$, then $y^2\ge 4xz$ for all $y\in[-M;M]$.
    \item If $4xz\ge 0$, then $y^2\ge 4xz$ if and only if $|y|\ge 2\sqrt{xz}$, so \\ $y\in [-M;-2\sqrt{xz}]\cup[2\sqrt{xz},M]$.
\end{itemize}

Since $f_A=f_B=f_C=\frac{1}{2M}$, we have $f_A(x)\cdot f_B(y)\cdot f_C(z)=\left(\frac{1}{2M}\right)^3=\frac{1}{8M^3}$.

The desired integral now becomes $\frac{1}{8M^3}\cdot(I+J)$, where
\[
I=\iint_{\substack{xz < 0 \\ x,z\in[-M;M]}}\int_{-M}^{M} 1\,\mathrm{d}y\,\mathrm{d}x\,\mathrm{d}z
\]
and
\[
J=\iint_{\substack{xz \ge 0 \\ x,z\in[-M;M]}}\int_{[-M;-2\sqrt{xz}]\cup[2\sqrt{xz};M]} 1\,\mathrm{d}y\,\mathrm{d}x\,\mathrm{d}z.
\]

\medskip
\noindent\textbf{Computing $I$:}

The innermost integral evaluates to $2M$. Then
\begin{align*}
I&=\iint_{\substack{xz < 0 \\ x,z\in[-M;M]}}2M\,\mathrm{d}x\,\mathrm{d}z\\
&=\int_{0}^M\int_{-M}^0 2M \,\mathrm{d}x\,\mathrm{d}z+ \int_{-M}^0\int_{0}^M 2M\,\mathrm{d}x\,\mathrm{d}z.
\end{align*}
Both of these are integrals of a constant function over a region of area $M^2$, so each evaluates to $2M^3$. Hence $I=4M^3$.

\medskip
\noindent\textbf{Computing $J$:}

The innermost integral is the length of the union $[-M;-2\sqrt{xz}]\cup[2\sqrt{xz};M]$, which is $2M-4\sqrt{xz}$. Thus
\begin{align*}
J&=\iint_{\substack{xz \ge 0 \\ x,z\in[-M;M]}}(2M-4\sqrt{xz})\,\mathrm{d}x\,\mathrm{d}z\\
&=\int_{0}^M\int_{0}^M (2M-4\sqrt{xz}) \,\mathrm{d}x\,\mathrm{d}z+ \int_{-M}^0\int_{-M}^0 (2M-4\sqrt{xz})\,\mathrm{d}x\,\mathrm{d}z.
\end{align*}
Note that both integrals have the same value by the substitution $x\mapsto -x$, $z\mapsto -z$. 
We get
\begin{align*}
J&=2\int_{0}^M\int_{0}^M (2M-4\sqrt{xz}) \,\mathrm{d}x\,\mathrm{d}z =2\left(M \cdot M\cdot 2M - 4\int_{0}^M\int_{0}^M\sqrt{xz}\,\mathrm{d}x\,\mathrm{d}z\right)\\
&=4M^3-8\left(\int_{0}^M\sqrt{t}\,\mathrm{d}t\right)^2 =4M^3-8\left( \frac{2}{3} t^{3/2} \bigg|_0^M \right)^2=4M^3-8\left(\frac{2}{3}M^{3/2}\right)^2\\
&=4M^3-8\cdot\frac{4}{9}M^3=4M^3-\frac{32}{9}M^3=\frac{4}{9}M^3.
\end{align*}

\subsection*{Final Answer}

Finally, we obtain
\[
\mathbb{P}[D\ge0]=\frac{1}{8M^3}\cdot(I+J)=\frac{1}{8M^3}\cdot\left(4M^3+\frac{4}{9}M^3\right)=\frac{1}{8M^3}\cdot\frac{40}{9}M^3=\frac{5}{9}.
\]
Since this probability does not depend on $M$, we have
\[
\lim_{M\to\infty} \mathbb{P}[D\ge0]=\boxed{\frac{5}{9}}.
\]

\end{document}